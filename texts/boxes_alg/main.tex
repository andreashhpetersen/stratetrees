\documentclass{article}

\usepackage[utf8]{inputenc}
\usepackage[margin=1.1in]{geometry}
\usepackage[parfill]{parskip}
\usepackage{algorithm}
\usepackage[noend]{algpseudocode}
\usepackage{amsmath}
\usepackage{amssymb}
\usepackage{graphicx} 
\usepackage[font=small]{caption}
\usepackage{subcaption} 

\graphicspath{{./imgs/}}


\title{Finding big boxes in partition strategy}
\author{Andreas Holck Høeg-Petersen}


\begin{document}

\maketitle

\section{Algorithm}

We start with a decision tree $\mathcal{T}$ representing af complete
partitioning of the state space. Let $\mathcal{V}$ be the set of variables in
the state space (ie.\ its dimensions) and let $\mathcal{L}$ be the set of all
leaves in the tree, where each leaf $L_{i}$ has a state $S_i =
\{(l_{i,1},u_{i,1}), \ldots, (l_{i,m},u_{i,m})\}$ consisting of a list of tuples
of lower and upper bounds for each of the $m$ variables in $\mathcal{V}$. This
state represents one $m$-dimensional box or area in the state space.

We then define $\mathcal{C}$ as the set of constraints in the partitioning given
by all the upper bounds, ie. $\mathcal{C} = \{ u_{1,1}, \ldots, u_{1,m},
u_{2,1}, \ldots, u_{n, m} \}$. Lastly, we define a mutable list $\mathcal{P}$ in
which we store all the points from which our searches for mergable partitions
begin. If we denote the minimum value of each variable $V_i \in V$ as
$v^{\min}_i$ then we initialise $\mathcal{P}$ as $\mathcal{P} = \{ (v^{\min}_1,
\ldots, v^{\min}_m) \}$.

The algorithm works by continually popping the next element of $\mathcal{P}$ and
using this point $p$ as a starting point. From $p$ we construct an
infinitesmally small box $b = \{ (p_1, p_1), \ldots, (p_m,p_m) \}$ and then
query $\mathcal{T}$ to find out what action $a_p$ is chosen at this symbolic
state (which at this point is the same as querying at point $p$). Then we pick
the constraint $u_{i,j} \in \mathcal{C}$ whose euclidean distance to the upper
bound of $b$ in dimension $j$ is the smallest and update $b$ to reflect this new
upper bound in dimension $i$. That is, we replace $(p_j, p_j) \in b$ with $(p_j,
u_{i,j})$.

Now we again query $\mathcal{T}$ for the action(s) in $b$ and continue updating
$b$ untill we get more than one type of action. We then know, that the latest
update of $b$ expanded the box into an area of the state space where a different
action is optimal, and thus we should not include it in our current box. We
therefore roll back the latest update and mark that particular dimension as
`exhausted'. If there are still unexhausted variables (dimensions) we continue
as before, but now discarding any constraints on the exhausted dimensions.

If after we have rolled back our update, we see that all variables are
exhausted, we then have a box $b$ that spans the largest possible area
(hyperplane?) from $p$ with only one optimal action. We store this and then add
to $\mathcal{P}$ the edge points of $b$. That is, if $b = \{ (l_{1},
u_{1}), \ldots, (l_{m}, u_{m}) \}$ then we let $\mathcal{P} = \mathcal{P} \cup
\{ (u_{1}, l_{2}, \ldots, l_{m}), (l_{1}, u_{2}, \ldots, l_{m}), \ldots, (l_{1},
l_{2}, \ldots, u_{m}) \}$. After that, we start over with the next point in
$\mathcal{P}$ and continue until we have emptied $\mathcal{P}$.

Below is the pseudocode for the algorithm which includes a couple of more checks
and edge cases needed for the algorithm to work. However, several details are
still left out or relegated to (hopefully) helpfully named functions.


\begin{algorithm}[ht]
    \caption{Find boxes}\label{alg:FindBoxes}

    \begin{algorithmic}[1]
        \Function{FindBoxes}{$\mathcal{T}$, $\mathcal{C}$, $\mathcal{V}$, $p_{start}$}

        \State{$boxes \gets \{\}$}
        \State{$\mathcal{P} \gets \{ p_{start} \} $}

            \item[]%
        \While{$\mathcal{P}$ is not empty}
            \State{$exhausted \gets \{\}$}
            \State{$\mathcal{P}.sort()$ } \Comment{left to right}
            \State{$p \gets \mathcal{P}.pop()$}

            \item[]%
            \If{$p$ has been visited}
                \Comment{can be checked by maintaining a separate tree}
                \State{continue}
            \EndIf%

            \item[]%
            \State{$b \gets \text{\Call{MakeBoxFromPoint}{$p$}}$}
            \State{\Call{SortAccordingToDistance}{$\mathcal{C}$, $p$}}
            \item[]%

            \For{$u_{i,j}$ in $\mathcal{C}$}
                \If{$u_{i,j} <= p_j$ or $V_j \in exhausted$}
                \Comment{$p_j$ is value of $V_j$ in $p$}
                    \State{continue}
                \EndIf%

                \item[]%
                \State{$oldVal \gets b^u_j$}
                \Comment{$b^u_j$ is the upper bound of variable $j$ in $b$}

                \State{$b^u_j \gets u_{i,j}$}

                \item[]%
                \If{\Call{Count}{$\mathcal{T}.actionsAtState(b)$} $>$ 1 \textbf{or}
                    $b$ is explored \textbf{or} $b^u_j = V^{\max}_j$ }

                    \State{$exhausted.add(V_j)$}

                    \If{$exhausted.size < \mathcal{V}.size$}
                        \State{continue}
                    \EndIf%

                    \item[]%
                    \If{$b^u_j < V^{\max}_j$}
                    \Comment{$V^{\max}_j$ is the maximum possible value of $V_j$ }
                        \State{$b^u_j \gets oldVal$}
                    \EndIf%

                    \State{$boxes.add(b)$}

                    \item[]%
                    \For{$p'$ in \Call{EdgePoints}{$b$}}
                        \State{$\mathcal{P}.add(p')$}
                    \EndFor%

                \EndIf%

            \EndFor%
        \EndWhile%

        \State{\textbf{return} $boxes$}

        \EndFunction%
    \end{algorithmic}

\end{algorithm}


\end{document}
