\section{Experiments}%
\label{sec:experiments}

We evaluate the effectiveness of \textsc{MaxPartitions} on two different kinds
of partitionings described in~\cref{sec:preliminaries}, namely safety shields
and controllers. For shields, we are interested in preserving the safety
guarantees of the input shield, while for controllers we are interested in
retaining performance as measured in mean reward.

\begin{table*}[!ht]
    \centering
    \begin{tabular}{lrcrcrc}
        \toprule
        \multirow{2}*{Model} & & & \multicolumn{2}{c}{\textsc{MaxPartitions}} &
        \multicolumn{2}{c}{\textsc{VIPER}}  \\
                             & Input size & Dimensions & Leaves & Unsafe runs &
                             Leaves & Unsafe runs \\
        \midrule
        Random walk            &    57,600 & 2 &     44 & \textbf{No} &    39 & \textbf{No} \\
        Cruise                 & 1,340,000 & 3 & 11,643 & \textbf{No} &    54 & Yes \\
        Oil pump               & 1,777,468 & 4 &    291 & \textbf{No} &   101 & Yes \\
        Bouncing ball          & 2,800,000 & 2 &  3,803 & \textbf{No} &    22 & Yes \\
        DCDC boost converter   & 6,994,242 & 3 &  7,600 & \textbf{No} & 1,392 & Yes \\
        \bottomrule
    \end{tabular}
    \caption{%
        Comparing \textsc{MaxPartitions} and \textsc{VIPER} for minimizing
        shields. The column `Unsafe runs' indicate wether a violation of the
        model-specific safety requirement were violated at least once during
        1000 simulations in a purposefully antagonistic environment.
    }\label{tab:shieldResults}
\end{table*}

In our experimental setup, we consider 5 different Reinforcement Learning
problems. For each of them, two versions of their environments are implemented,
one in \textsc{UPPAAL Stratego} which we use to learn near-optimal control
strategies, and one in \textsc{Gymnasium}~\cite{towersGymnasium2023} which we
use for synthesizing safety shields. Further, since we want to compare
\textsc{MaxPartitions} with \textsc{VIPER}, we train our \textsc{VIPER} models
in the \textsc{Gymnasium} environments using the \textsc{UPPAAL Stratego}
policies as oracles. For finding minimal safety shields with \textsc{VIPER}, we
provide the synthesized shield as a non-deterministic controller and evaluate
wether the resulting \textsc{VIPER} controller maintains the safety properties.

\cref{tab:shieldResults} shows the results for minimizing shields. Both
\textsc{MaxPartitions} and \textsc{VIPER} achieve substantial reductions
compared to the original input. However, for all but one example, the
\textsc{VIPER} minimized shield encounters one or more unsafe runs through 1000
simulations of the environment. The shields minimized with
\textsc{MaxPartitions} does not, which is expected since \textsc{MaxPartitions}
preserves an equivalent mapping to that of the input shield, which is designed
to be safe. The largest reduction that still retains safety is achieved by
\textsc{MaxPartitions} for the {\tt Oil pump} example, where the minimized
shield has a size that is only 0.01\% of the original.

When we turn our attention to control strategies
in~\cref{tab:controllerResults}, we also see large reductions for both
\textsc{MaxPartitions} and \textsc{VIPER}, though once again with a clear
advantage in \textsc{VIPER}. Both models retain the same performance, which is
not surprising since \textsc{MaxPartitions} preserve the behavior of the
original strategy and \textsc{VIPER} is specifically designed to adopt the
important dynamics of an oracle strategy. The difference in size can be
explained by this specific nature of the two methods: whereas
\textsc{MaxPartitions} deliberately produces an equivalent state-action mapping
to that of the input strategy, \textsc{VIPER} discards `irrelevant' parts of the
strategy that do not appear during the execution of the environment or where the
choice of action is not relevant. This allows \textsc{VIPER} to prune many more
decisions than \textsc{MaxPartitions}, resulting in a smaller output
partitioning.

\begin{table*}[!hb]
    \centering
    \begin{tabular}{lrcrcrc}
        \toprule
        \multirow{2}*{Model} & & & \multicolumn{2}{c}{\textsc{MaxPartitions}} &
        \multicolumn{2}{c}{\textsc{VIPER}}  \\
                             & Input size & Dimensions & Leaves & Mean reward &
                             Leaves & Mean reward \\
        \midrule
        Random walk            &    
               75 & 2 &    25 &  -18.2 (+/-  0.98) &  12 &   -18.9 (+/-  1.44) \\

        Cruise                 &
            3,175 & 3 & 1,120 & -722.8 (+/- 426.8) &  70 & -340.52 (+/- 432.5) \\

        Oil pump               &
              --- & 4 &   --- &                --- & --- &                 --- \\

        Bouncing ball          &
            3,609 & 2 &   184 &  -36.3 (+/-  3.2) &  31 &    -36.3 (+/- 2.8) \\

        DCDC boost converter   &
            5,225 & 3 &   681 &   -3.9 (+/-  1.5) &  60 &     -3.9 (+/-  1.7) \\
        \bottomrule
    \end{tabular}
    \caption{%
        Comparing \textsc{MaxPartitions} and \textsc{VIPER} for minimizing
        controllers.
    }\label{tab:controllerResults}
\end{table*}

These results encourages a combination of the two methods, when dealing with
safety critical systems: using \textsc{MaxPartitions} for minimizing a safety
shield and imposing it on a near-optimal controller minimized with
\textsc{VIPER}. In~\cref{tab:combinedResults} we show that with this
combination, we are able to obtain safe and near-optimal controllers, that have
been minimized from much larger originals, when we use a combination of these
methods.

\begin{table*}[!ht]
    \centering
    \begin{tabular}{lrcrcrc}
        \toprule
        \multirow{2}*{Model} & & & \multicolumn{2}{c}{\textsc{MaxPartitions}} &
        \multicolumn{2}{c}{\textsc{VIPER}}  \\
                             & Input size & Dimensions & Leaves & Mean reward &
                             Leaves & Mean reward \\
        \midrule
        Random walk            &    
               75 & 2 &    25 &  -18.2 (+/-  0.98) &  12 &   -18.9 (+/-  1.44) \\

        Cruise                 &
            3,175 & 3 & 1,120 & -722.8 (+/- 426.8) &  70 & -340.52 (+/- 432.5) \\

        Oil pump               &
              --- & 4 &   --- &                --- & --- &                 --- \\

        Bouncing ball          &
            3,609 & 2 &   184 &  -36.3 (+/-  3.2) &  31 &    -36.3 (+/- 2.8) \\

        DCDC boost converter   &
            5,225 & 3 &   681 &   -3.9 (+/-  1.5) &  60 &     -3.9 (+/-  1.7) \\
        \bottomrule
    \end{tabular}
    \caption{%
        Combining \textsc{MaxPartitions} and \textsc{VIPER}.
    }\label{tab:combinedResults}
\end{table*}
