\section{Preliminaries}%
\label{sec:preliminaries}

We start by defining some relevant concepts.

% \begin{definition}[Strategy]%
%     A strategy $\sigma : S \to A$ over the state space $\mathcal{S}$ is a
%     mapping from any state $s \in \mathcal{S}$ to an action $a \in Act$ where
%     $Act$ is a finite set of allowed actions. 
% \end{definition}

% A strategy can be represented in several different ways. In
% Q-learning~\cite{Sutton1998}, the approach is to estimate a function $Q : S, A
% \to \mathbb{R}$ that maps a state-action pair $(s,a)$ to a real number, that is
% an estimation of the expected cost of taking action $a$ in state
% $s$.\footnote{The Q-value can also be the expected reward.} Choosing an action
% in state $s$ under the strategy $\sigma$ is then given by $\sigma(s) =
% \argmin_{a \in Act}Q(s,a)$.

% For a continuous state space, the Q-function can either be estimated using
% function approxmiation techniques or by discretization of the state space. In
% the latter case, the state space is partitioned into a finite set of convex
% regions, each defining a discrete state $S$ as a list of lower and upper bounds,
% $S_i^{\min}, S_i^{\max}$, for each $i = 1,\ldots,K$. The Q-values can then be
% estimated for the entire region $S$, and we say that $Q(s,a) = Q(S,a)$ for all
% $s \in S$. This allows for a tabular representation of the Q-function, as in the
% toy example in Table~\ref{tab:exStrategyQTable}.

\begin{definition}[Partitions]%
    A partitioning $\mathcal{A}$ of the state space $\mathcal{S} \in
    \mathbb{R}^K$ is a set of regions $\nu$ that divides $\mathcal{S}$ such that
    $\bigcup_{\nu \in \mathcal{A}}\nu = \mathcal{S}$ and for any two regions
    $\nu, \nu' \in \mathcal{A}$ where $\nu \neq \nu'$ it holds that $\nu \cap
    \nu' = \emptyset$. Each region $\nu$ can be expressed in terms of two
    points, $s^{\min}$ and $s^{\max}$, so that for each $s = (s_1, \ldots, s_K)
    \in \nu$ it holds that $s^{\min}_i < s_i \le  s^{\max}_i$ for $i =
    1,\ldots,K$.
\end{definition}

Evidently, any discretization of a state space $\mathcal{S} \in \mathbb{R}^K$ is
effectively a partitioning. For example, the Q-table in
Table~\ref{tab:exStrategyQTable} corresponds to the partitioning $\mathcal{A} =
\{ ((0,0),(1,1)), ((1,0),(2,1)),\ldots, ((1,2),(2,3)), ((2,2),(3,3)) \}$.

\begin{definition}[Decision tree]%
\label{def:decisionTree}
    A binary decision tree over the domain $\mathcal{S} \in \mathbb{R}^K$ is a
    tuple $\mathcal{T} = (\eta_{0}, \mathcal{N}, \mathcal{L})$ where $\eta_{0}
    \in \mathcal{N}$ is the root node of the tree, $\mathcal{N}$ is a set of
    branching nodes and $\mathcal{L}$ is a set of leaf nodes. Each branch node
    $\eta \in \mathcal{N}$ consists of two child nodes and a predicate function 
    of the form $\rho(s) = s_{i} \leq c$ with $s \in \mathcal{S}$ and $c$ being a
    constant. Each leaf node $\ell \in \mathcal{L}$ is assigned a label from a
    set of labels $\mathcal{U}$.
\end{definition}

For a decision tree $\mathcal{T}$, we can obtain a decision $\delta =
\mathcal{T}(s)$, $\delta \in \mathcal{U}$ from any state $s \in \mathcal{S}$ by
following the \textit{path} from the root node to a leaf by evaluating $\rho(s)$
at every branch node and continuing the path at the left child if the predicate
is true and at the right child otherwise. Further, we also allow evaluating a
region of $\mathcal{S}$.  Given a region $\nu = (s^{\min}, s^{\max})$,
$[\delta]_{\nu} = \mathcal{T}(\nu)$ is the set of all decisions that can be
obtained evaluating configurations of $\nu$, ie. $\mathcal{T}(\nu) = \{
\mathcal{T}(s) \mid s \in \nu \}$.

For any node $\eta$ in the tree (branching or leaf) the predicates along the
path to this node defines a region $\nu$ and we denote this as $\nu =
\lambda(\eta)$. In this way, we say that $\mathcal{T}$ induces a partitioning
$\mathcal{A}_{\mathcal{T}} = \{ \lambda(\ell) \mid \ell \in \mathcal{L} \}$. For
any region $\nu$ and a decision tree $\mathcal{T}$ we say, that $\nu$ has
\textit{singular mapping} in $\mathcal{T}$ if for all $p \in \nu$,
$\mathcal{T}(p) = \delta$ for some $\delta \in \mathcal{U}$. Naturally, all
regions in $\mathcal{A}_{\mathcal{T}}$ has singular mapping in $\mathcal{T}$.
For any partitioning $\mathcal{B}$ of the same state space, we say $\mathcal{B}$
\textit{respects} $\mathcal{T}$ if and only if every region $\nu \in
\mathcal{B}$ has singular mapping in $\mathcal{T}$.
